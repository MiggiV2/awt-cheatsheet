\lstset{language=TypeScript}
\begin{center}
    {\Huge \textbf{AWT Cheat Sheet}}
\end{center}

\begin{multicols*}{3}

% ================= SECTION 1 =================
\section{Angular Components \& Basics}

\subsection{Component Structure}
A component consists of a Template (HTML), Style (CSS), and Class (TS).
\begin{lstlisting}
@Component({
  selector: 'app-hello', // Usage: <app-hello>
  template: `<h1>Hello {{name}}</h1>`,
  styles: [`h1 { font-weight: normal; }`]
})
export class HelloComponent implements OnInit {
  name: string = 'World';
  
  constructor() { } // DI Injection here
  
  ngOnInit(): void {
    // Init logic (API calls)
  }
}
\end{lstlisting}

\subsection{Lifecycle Hooks}
\begin{itemize}[leftmargin=*]
    \item \texttt{ngOnChanges}: Input property changes.
    \item \texttt{ngOnInit}: Comp. initialized (best for data fetch).
    \item \texttt{ngAfterViewInit}: View fully initialized.
    \item \texttt{ngOnDestroy}: Cleanup (subscriptions).
\end{itemize}

% ================= SECTION 2 =================
\section{Data Binding}

\subsection{Types of Binding}
\begin{itemize}[leftmargin=*]
    \item \textbf{Interpolation} (TS $\to$ DOM): \\
    \texttt{Welcome \{\{ name \}\}}
    \item \textbf{Property Binding} (TS $\to$ DOM): \\
    \texttt{<img [src]="imageUrl">} \\
    \texttt{<button [disabled]="!isValid">}
    \item \textbf{Event Binding} (DOM $\to$ TS): \\
    \texttt{<button (click)="save()">Save</button>}
    \item \textbf{Two-Way Binding} (TS $\leftrightarrow$ DOM): \\
    Requires \texttt{FormsModule} import. \\
    \texttt{<input [(ngModel)]="username">}
\end{itemize}

% ================= SECTION 3 =================
\section{Control Flow (Angular 17+)}

\subsection{Conditional (@if)}
Replaces \texttt{*ngIf}.
\begin{lstlisting}
@if (isLoggedIn) {
  <user-dashboard />
} @else if (isLoading) {
  <spinner />
} @else {
  <login-form />
}
\end{lstlisting}

\subsection{Loops (@for)}
Replaces \texttt{*ngFor}. Requires \texttt{track}.
\begin{lstlisting}
<ul>
  @for (user of users; track user.id; let i = $index) {
    <li>{{ i }}: {{ user.name }}</li>
  } @empty {
    <li>No users found.</li>
  }
</ul>
\end{lstlisting}

\subsection{Switch (@switch)}
\begin{lstlisting}
@switch (status) {
  @case ('active') { <span class="green">OK</span> }
  @case ('banned') { <span class="red">NO</span> }
  @default { <span>Unknown</span> }
}
\end{lstlisting}

% ================= SECTION 4 =================
\section{Services \& DI}

\subsection{Defining a Service}
Singleton services handle business logic/HTTP.
\begin{lstlisting}
@Injectable({ providedIn: 'root' })
export class DataService {
  constructor(private http: HttpClient) {}

  getData(): Observable<Data[]> {
    return this.http.get<Data[]>('/api/items');
  }
}
\end{lstlisting}

\subsection{Injecting \& Using Service}
\begin{lstlisting}
export class MyComponent {
  // Injection via Constructor
  constructor(private dataService: DataService) {}

  load() {
    this.dataService.getData().subscribe({
      next: (data) => this.items = data,
      error: (err) => console.error(err)
    });
  }
}
\end{lstlisting}

% ================= SECTION 5 =================
\section{Forms}

\subsection{Template-Driven (Simple)}
Best for simple forms (Login demo).
\begin{itemize}[leftmargin=*]
    \item Import \texttt{FormsModule}.
    \item Use \texttt{[(ngModel)]} for binding.
    \item Use \texttt{\#ref="ngModel"} for validation state.
\end{itemize}
\begin{lstlisting}
<form #f="ngForm" (ngSubmit)="login(f)">
  <input name="user" [(ngModel)]="user.name" 
         required #u="ngModel">
  
  @if (u.invalid && u.touched) {
    <div>Name required</div>
  }
  
  <button [disabled]="f.invalid">Login</button>
</form>
\end{lstlisting}

\subsection{Reactive Forms (Complex)}
More control, testable. (ACME demo).
\begin{itemize}[leftmargin=*]
    \item Import \texttt{ReactiveFormsModule}.
    \item Logic in Class (\texttt{FormGroup}, \texttt{FormControl}).
\end{itemize}
\begin{lstlisting}
// Component
form = new FormGroup({
  name: new FormControl('', Validators.required),
  email: new FormControl('', [Validators.email])
});

// Template
<form [formGroup]="form" (ngSubmit)="save()">
  <input formControlName="name">
</form>
\end{lstlisting}

% ================= SECTION 6 =================
\section{Routing}

\subsection{Route Definition (routes.ts)}
\begin{lstlisting}
export const routes: Routes = [
  { path: '', redirectTo: 'home', pathMatch: 'full' },
  { path: 'home', component: HomeComponent },
  { path: 'users/:id', component: UserDetailComponent },
  { path: '**', component: NotFoundComponent }
];
\end{lstlisting}

\subsection{Navigation}
\textbf{In Template:}
\begin{lstlisting}
<a [routerLink]="['/users', user.id]">View</a>
<router-outlet></router-outlet> <!-- Placeholder -->
\end{lstlisting}

\textbf{In Code:}
\begin{lstlisting}
constructor(private router: Router) {}

goBack() {
  this.router.navigate(['/home']);
}
\end{lstlisting}

% ================= SECTION 7 =================
\section{Observables (RxJS)}

\subsection{Key Concepts}
\begin{itemize}[leftmargin=*]
    \item \textbf{Observable}: Stream of events over time.
    \item \textbf{Observer}: Object with \texttt{next}, \texttt{error}, \texttt{complete}.
    \item \textbf{Subscription}: Execution of an Observable.
\end{itemize}

\subsection{Common Operators}
\begin{lstlisting}
import { map, filter, tap } from 'rxjs/operators';

this.http.get('/api/users').pipe(
  tap(data => console.log('Raw:', data)), // Side effect
  filter(user => user.isActive),          // Filter
  map(user => user.fullName)              // Transform
).subscribe(names => this.names = names);
\end{lstlisting}

\subsection{Async Pipe (Best Practice)}
Avoid manual \texttt{.subscribe()} in components to prevent memory leaks.
\begin{lstlisting}
// TS
users$ = this.service.getUsers();

// HTML
@for (user of users$ | async; track user.id) { ... }
\end{lstlisting}

% ================= SECTION 8 =================
\section{Common Commands (CLI)}

\begin{description}[font=\ttfamily\bfseries]
\item[ng new app] Create new project
\item[ng serve] Run dev server (port 4200)
\item[ng g c name] Generate Component
\item[ng g s name] Generate Service
\item[ng g m name] Generate Module
\item[ng build] Build for production
\end{description}

\section{Modules (@NgModule)}
\textit{Note: Only if not using Standalone Components.}
\begin{lstlisting}
@NgModule({
  declarations: [AppComponent, HelloComponent],
  imports: [
    BrowserModule, 
    FormsModule, 
    HttpClientModule
  ],
  providers: [DataService],
  bootstrap: [AppComponent]
})
export class AppModule { }
\end{lstlisting}

\section{Important Imports}
\begin{lstlisting}
import { Component, OnInit } from '@angular/core';
import { CommonModule } from '@angular/common';
import { FormsModule } from '@angular/forms';
import { Router } from '@angular/router';
import { HttpClient } from '@angular/common/http';
\end{lstlisting}

\section{Troubleshooting}
\begin{itemize}[leftmargin=*]
    \item \textbf{No provider for X:} Add X to \texttt{providers} array or use \texttt{providedIn: 'root'}.
    \item \textbf{Can't bind to 'ngModel':} Import \texttt{FormsModule}.
    \item \textbf{404 on Refresh:} Server must redirect all routes to \texttt{index.html}.
    \item \textbf{ExpressionChangedAfter...:} Data changed during change detection cycle. Use \texttt{ChangeDetectorRef}.
\end{itemize}

\end{multicols*}
